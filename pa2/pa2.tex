\documentclass[11pt]{article}
\usepackage{fullpage}
\usepackage{geometry}                % See geometry.pdf to learn the layout options. There are lots.
\geometry{letterpaper}                   % ... or a4paper or a5paper or ... 
%\geometry{landscape}                % Activate for for rotated page geometry
%\usepackage[parfill]{parskip}    % Activate to begin paragraphs with an empty line rather than an indent
\usepackage{graphicx}
\usepackage{amssymb}
\usepackage{amsmath}
\usepackage{algorithmic}
\usepackage{epstopdf}
\DeclareGraphicsRule{.tif}{png}{.png}{`convert #1 `dirname #1`/`basename #1 .tif`.png}
\usepackage{enumerate}

\title{CS E124 Programming Assignment 2}
\author{Stephen Karger}
\date{March 31, 2012}                                           % Activate to display a given date or no date

\begin{document}
\maketitle

\section{Analysis of optimal cross-over point}
Assuming that the cost of any single arithmetic operation (adding, subtracting, multiplying, or dividing two real numbers) is 1, the following equations model conventional matrix multiplication and Strassen's algorithm.
\begin{flalign*}
\text{Conventional: } & 2n^{3} - n^{2} & \\
\text{Strassen: } & 6n^{2 + log(7/4)} + n^{log\,7} - 6n^{2} &
\end{flalign*}
Derivation: \\
For each individual entry in the matrix product, the conventional algorithm performs $n$ multiplications. Then it takes $n-1$ additions to combine them, for $2n - 1$ total operations. Since there are $n^2$ elements to compute, the number of operations for the entire algorithm is $2n^3 - n^2$. \\
\\
In Strassen's algorithm multiplying an $n$ x $n$ matrix involves 10 additions or subtractions of dimension $n/2$ matrices to compute $P_1$ through $P_7$, then 7 multiplications of dimension $n/2$ matrices using $P_1 \dots P_7$, and finally 8 additions of dimension $n/2$ matrices to combine the results. Adding matrices of dimension $n/2$ requires $(\frac{n}{2})^2$ real number additions, so a recurrence relation for the number of multiplications is $T(n) = 18(\frac{n}{2})^2 + 7T(n/2)$. \\
\\
Clearly $T(1) = 1$. Expanding the recurrence relations yields
\begin{flalign*}
T(n) &= 18 \left (\frac{n}{2} \right )^2 + 7T(n/2) = 18 \left (\frac{n}{2} \right )^2 + 7\ \left [  18 \left (\frac{n}{4} \right )^2 + 7T(n/4)  \right ] & \\
&= 18 \left (\frac{n}{2} \right )^2 + 7\ \left [  18 \left (\frac{n}{4} \right )^2 + 7 \left [ 18 \left (\frac{n}{8} \right )^{2} + 7T(n/8)  \right ]  \right ] & \\
&= 18 \left (\frac{n}{2} \right )^2 + 7*18 \left (\frac{n}{4} \right )^2 + 7^2*18 \left (\frac{n}{8} \right )^{2} + \cdots + 7^{log\,n - 1} \left (18 \left ( \frac{n}{2^{log\,n}} \right )^{2} + 7  \left ( \frac{n}{2^{log\,n}} \right )^{2} \right ) & \\
&= 7^{log\,n} + \sum_{0}^{log\,n - 1} 7^i * 18 \left (\frac{n}{2^{i+1}} \right )^2 = 7^{log\,n} + 18 \left (\frac{n}{2} \right )^2 \sum_{0}^{log\,n - 1} \left ( \frac{7}{4} \right )^{i}
\end{flalign*}
\\
Using the geometric series formula for the sum above gives
\begin{flalign*}
T(n) &= 7^{log\,n} + 18 \left (\frac{n}{2} \right )^2 \sum_{0}^{log\,n - 1} \left ( \frac{7}{4} \right )^{i} = 7^{log\,n} + 18 \left ( \frac{n}{2} \right )^2 * \frac{1 - \left ( \frac{7}{4} \right )^{log\,n} } {1 - \frac{7}{4} }
\end{flalign*}
Finally, using the fact that $a^{log\,b} = b^{log\,a}$ and simplifying algebraically 
\begin{flalign*}
T(n) &= n^{log\,7} + 18 \left ( \frac{n}{2} \right )^2 * \frac{1 - n^{log \left ( \frac{7}{4} \right )} } { 1 - \left ( \frac{7}{4} \right ) } =
n^{log\,7} + 18 \left ( \frac{n}{2} \right )^2 * \frac{n^{log \left ( \frac{7}{4} \right )}} { \frac{7}{4} - 1 } \\
&= n^{log\,7} + 18 \left ( \frac{n^{2}}{4} \right ) * \frac{ n^{log \left ( \frac{7}{4} \right )} - 1} { \frac{3}{4} } = 
n^{log\,7} + 6n^{2} \left ( n^{log \left ( \frac{7}{4} \right )} - 1 \right ) \\
&= n^{log\,7} + 6n^{2 + log \left ( \frac{7}{4} \right )} - 6n^2 \\
&= 6n^{2 + log(7/4)} + n^{log\,7} - 6n^{2}
\end{flalign*}
Comparing these two functions, Strassen's algorithm requires more operations than the conventional method until the matrices have dimension 655, so the estimate for $n_0 = 655$. \\

\begin{tabular}{l|l|l}
n & Conventional & Strassen \\  \hline
650 & 548827500 & 549471506 \\
651 & 551365101 & 551851137 \\
652 & 553910512 & 554237392 \\
653 & 556463745 & 556630282 \\
654 & 559024812 & 559029813 \\
655 & 561593725 & 561435994 \\
656 & 564170496 & 563848833 \\
657 & 566755137 & 566268339 \\
658 & 569347660 & 568694520 \\
659 & 571948077 & 571127384 \\
660 & 574556400 & 573566938
\end{tabular}

\end{document} 